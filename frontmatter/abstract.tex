%The word �Abstract� should be centered 2? below the top of the page. 
%Skip one line, then center your name followed by the title of the 
%thesis/dissertation. Use as many lines as necessary. Centered below the 
%title include the phrase, in parentheses, �(Under the direction of  
%_________)� and include the name(s) of the dissertation advisor(s).
%Skip one line and begin the content of the abstract. It should be 
%double-spaced and conform to margin guidelines. An abstract should not 
%exceed 150 words for a thesis and 350 words for a dissertation. The 
%latter is a requirement of both the Graduate School and UMI's 
%Dissertation Abstracts International.
%Because your dissertation abstract will be published, please prepare and 
%proofread it carefully. Print all symbols and foreign words clearly and 
%accurately to avoid errors or delays. Make sure that the title given at 
%the top of the abstract has the same wording as the title shown on your 
%title page. Avoid mathematical formulas, diagrams, and other 
%illustrative materials, and only offer the briefest possible description 
%of your thesis/dissertation and a concise summary of its conclusions. Do 
%not include lengthy explanations and opinions.
%The abstract should bear the lower case Roman number ii (if you did not 
%include a copyright page) or iii (if you include a copyright page).

\begin{center}
\vspace*{52pt}
{ABSTRACT}
\vspace{11pt}

\begin{singlespace}
Eric Parajon: Framing the Super Wicked Problem: Three Essays on American Climate Attitudes \\
(Under the direction of Cameron Ballard-Rosa)
\end{singlespace}
\end{center}


Addressing global climate change effectively and equitably requires significant political will and public support. In the United States, persistent climate skepticism complicates this effort. This dissertation investigates the political psychology of climate attitudes, with a focus on how identity-based factors, particularly racial resentment and nationalism, influence support for climate action.

This dissertation consists of three papers. The first paper examines how the perceived racial distributional effects of climate policy shape the views of White Americans. I argue that racial resentment has emerged as a key determinant of climate attitudes, especially when policies are seen as benefiting communities of color. Using both correlational and experimental survey data, I find that higher levels of racial resentment are associated with lower support for domestic and international climate policy, regardless of partisanship. An original survey experiment shows that support declines further when respondents are explicitly informed about the racial equity goals of climate action.

The second paper investigates whether racial resentment operates as a distinct mechanism or merely reflects broader conservative worldviews. Using an original survey of White Americans, I test competing explanations for the linkage finding that racial resentment remains a significant and independent predictor of opposition to climate policy, even after accounting for other psychological and ideological factors. These results indicate that racial resentment is not merely a proxy for cultural conservatism, but represents a distinct form of out-group animus that actively shapes climate attitudes.

The third paper, co-authored with Tyler Ditmore, explores strategies to overcome climate skepticism. Through a vignette and a conjoint experiment, we find that framing green industrial policy as a tool of international economic competition, particularly with China, is especially effective in shifting opinions among ex-ante climate skeptics.


Collectively, these papers highlight the importance of non-material factors, such as racial attitudes and sociotropic concerns, in shaping public support for climate policy. They offer critical insights for scholars and policymakers on how climate messaging can either build or erode coalitions for action. Understanding these dynamics is essential for crafting politically viable and inclusive climate strategies.



\clearpage
